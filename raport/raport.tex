\documentclass[12pt,a4paper]{article}
\usepackage[utf8]{inputenc}
\usepackage[T1]{fontenc}
\usepackage{amsfonts}
\usepackage{amsmath}
\usepackage{setspace}
\usepackage{wrapfig}
\usepackage[english]{babel}
\usepackage{graphicx}
\usepackage{caption}
\usepackage{amssymb}
\usepackage{changepage}
\usepackage{listings}

\title{Sztuczna inteligencja i inżynieria wiedzy, \\ Lista 1}
\author{Tymoteusz Trętowicz, 260451}
\date{}

\usepackage{xcolor}

\definecolor{codegreen}{rgb}{0,0.6,0}
\definecolor{codegray}{rgb}{0.5,0.5,0.5}
\definecolor{codepurple}{rgb}{0.58,0,0.82}
\definecolor{backcolour}{rgb}{0.95,0.95,0.92}

\lstdefinestyle{mystyle}{
    backgroundcolor=\color{backcolour},
    commentstyle=\color{codegreen},
    keywordstyle=\color{magenta},
    numberstyle=\tiny\color{codegray},
    stringstyle=\color{codepurple},
    basicstyle=\ttfamily\footnotesize,
    breakatwhitespace=false,
    breaklines=true,
    captionpos=b,
    keepspaces=true,
    numbers=left,
    numbersep=5pt,
    showspaces=false,
    showstringspaces=false,
    showtabs=false,
    tabsize=2
}

\lstset{style=mystyle}


\setstretch{1.1}

\begin{document}
    \maketitle
    \pagebreak
    \section*{Implementacjia}
    Implementacja grafu opiera się na słowniku nazw przystanków do ich wartości.
\begin{lstlisting}[language=Python]
class Node:
  def __init__(self, name, x, y):
    self.name = name
    self.x = float(x)
    self.y = float(y)
    self.edges = {}
\end{lstlisting}

    W każdym wierzchołku (obiekcie klasy Node) zapisujemy informacje o nazwie, wysokości i szerokości geograficznej, oraz o wychodzących z tego wierzchołka krawędziach (połączeniach do innych przystanków). \texttt{edges} jest strukturą danych w postaci zagnieżdżonych słowników: jest to mapa nazwy przystanku do mapy godzin odjazdu do tego przystanku do informacji o połączeniu. Przykładowo:
    \begin{lstlisting}[language=Python]

    graph['Rynek'].edges['Zamkowa']['12:52:30']
    \end{lstlisting}
    Jeżeli słownik obiektów kalsy Node nazywałby się \texttt{graph} to powyższe wyrażenie by zwróciło informacje o połączeniu z przystanku Rynek, do przystanku Zamkowa o godzinie 12:52:00. Te informacje to czas odjazdu z przystanku startowego, czas dojazdu do przystanku docelowego i nazwa lub numer lini.

    \pagebreak
    \subsubsection*{Algorytm Dijkstry}
    \noindent
    Zaimplementowany algorytm Dijkstry można przedstawić w postaci listy kroków:
    dla przystanku początekowego, przystanku docelowego i godziny startowej:
    \begin{enumerate}
        \item Utwórz pustą kolejkę przystanków do odwiedzenia.
        \item Dodaj przystanek początkowy, oraz godzinę znalezienia się na tym przystanku do kolejki.
        \item Pobierz \emph{obecny przystanek}, oraz ścieżkę z przystanku początekowego do obecnego przystanku z początku kolejki oraz usuń ten element z kolejki.
        \item Jeżeli \emph{obecny przystanek} jest przystnakiem końcowym zwróć scieżkę do obecnego przystanku.
        \item Dla każdego połączenia wychodzącego z tego przystanku: \begin{enumerate}
            \item Jeżeli godzina odjazdu do \textbf{następnego przystanku} już minęła, lub następny przystanek był już odwiedzony, pomiń to wychodzące połączenie.
            \item Policz koszt tego połączenia jako: $$c_{\text{dojazdu do \textbf{następnego przystanku}}} = c_{\text{doajzdu do \emph{obecnego przystanku}}} + c_{\text{połączenia}}$$
            \item Dodaj \textbf{następny przystanek} oraz to połączenie do kolejki przystnaków do odwiedzenia.
            \item Jezeli policzony koszt jest niższy niż obecny zapamiętany zapamiętaj nowy koszt jako najniższy oraz obecną scieżkę do \textbf{następnego przystanku}.
        \end{enumerate}
        \item Oznacz \emph{obecny przystanek} jako odwiedzony i wróć do punktu 3.
    \end{enumerate}
    Implementacja alogrytmu poza wymienionymi paramterami akceptuje jeszcze funkcję kosztu $f$ taką, że $$f(\text{obecny przystanek, następny przystanek, ...}) = c_{\text{połączenia}}$$
    \subsection*{Kryteria}
    \subsubsection*{Kryetium czasu}
    Kryterium czasu jest rozumiane jako funkcja minimalizująca czas spędzony na przystanku czekając. Dla danego przystanku oraz odjeżdząjących z nich tramwajów (lub autobusów) jest ono liczone jako:$$c_{\text{połączenia}} = t_{\text{godzina odjazdu z przystanku}} - t_{\text{teraz}}$$.
    \subsubsection*{Kryetium przesiadek}
    Kryetium przesiadek jest rozumiane jako funkcja minimalizująca ilość przesiadek na połączeniu od przytanku startowego do przystanku końcowego mając nadal na uwadze kryetium czasu:
    \begin{align*}
        c_{\text{połączenia}}  &= 10 \cdot i_{\text{ilość przysiadek}} + 0.3 \cdot  c_{\text{kryetium czasu}}\\
        &= 10 \cdot i_{\text{ilość przysiadek}} + 0.3 \cdot  (t_{\text{godzina odjazdu z przystanku}} - t_{\text{teraz}})\\
    \end{align*}
    Kryterium przesiadek uwzględnia kryterium czasu z wagą $0.3$ (arbitralny wybór). Jeżeli waga dla kryterium czasu byłaby równa $1$, znaczyłoby to że przesiadka na inną linię jest równoważna z czekaniem 10 minut na przystanku.
    \subsubsection*{Kryterium czasu z heurytyką odległości Manhattan}
    Kryerium czasu wzbogacone o heurystyke odległości Manhattan stara się zminimalizować czas przejazdu, mając na uwadzę, jak zmienia się oldegłość od naszego celu z danego przystanku. Jest ono wyliczane jako:
    \begin{align*}
        c_{\text{połączenia}} =& K_{\text{odległość Manhattan}} + \frac{c_{\text{kryterium czasu}}}{2}\\
        =& \text{ } |p_{\text{szerokość obecnego przystanku}} - p_{\text{szerokość końcowego przystanku}} |\\
        & + |p_{\text{długość obecnego przystanku}} - p_{\text{długość końcowego przystanku}} | \\
        & + \frac{c_{\text{kryterium czasu}}}{2}
    \end{align*}
    Tutaj oldegłość Manhattan jest liczona jako suma wartości bezwzględnej różnicy odpowiednio szerokości i długości geograficznej przystanku obecnego i przystanku docelowego.
    \subsubsection*{Kryterium czsu i przesiadek z heurytyką odległości Manhattan}
    Kryerium czasu i przesiadek z heurytyką odległości Manhattan jest liczone wykorzystując wyżej zdefiniowane kryteria:
    $$c_{\text{połączenia}} = K_{\text{odległość Manhattan}} + \frac{c_{\text{kryterium przesiadek}}}{2}$$
    \subsection*{Przykład}
    Przykładowe wywołanie oraz użycie implementacji:
\begin{lstlisting}[language=Python]
    START = "Grota-Roweckiego"
    STOP = "most Grunwaldzki"
    TIME = "17:27:00"

    pipe("./connection_graph.csv",
        read_file,
        spawn_graph,
        lambda graph: measure_time(lambda:
            shortest_path(graph,
                          [START, STOP],
                          TIME,
                          cost_time_dist)
        ),
        print_path
    )
\end{lstlisting}

    Powyższy fragment kodu, odczytuje dane z pliku \texttt{connection\_graph.csv} z lokalnego folderu, tworzy na ich podstawie graf, oraz uruchamia funkcję znajdującą najkrótszą ścieżkę. Przystanek startowy został ustawiony jako Grota-Roweckiego, przystanek docelowy to most Grunwaldzki. Czas startowy to 17:27:00. Funkcja \texttt{shortest\_path} akceptuje również funkcję kosztu jako paramter. Tutaj została podana funkcja licząca koszt wg. kryterium czasu z heurytyką odległości Manhattan.

    Wynikiem takiego wywołania jest:

    TIME USED:  0.071726044989191

% \begin{center}
\noindent
\begin{adjustwidth}{-40pt}{100pt}
\begin{tabular}{ |c|c|c|c|c| }
\hline
Przystanek początkowy & Przystanek końcowy &Odjazd&Przyjazd&Linia\\
\hline
Grota-Roweckiego & Gałczyńskiego & 17:29:00 & 17:31:00 & 113\\
Gałczyńskiego & Oboźna & 17:31:00 & 17:32:00 & 113 \\
Oboźna & Parafialna & 17:32:00 & 17:33:00 & 113 \\
Parafialna & Wojszyce & 17:33:00 & 17:34:00 & 113 \\
Wojszyce & Przystankowa & 17:34:00 & 17:36:00 & 113 \\
Przystankowa & Borowska (Szpital) & 17:36:00 & 17:37:00 & 113 \\
Borowska (Szpital) & Działkowa & 17:37:00 & 17:39:00 & 113 \\
Działkowa & ROD Bajki & 17:39:00 & 17:41:00 & 143 \\
ROD Bajki & Śliczna & 17:41:00 & 17:42:00 & 143 \\
Śliczna & Borowska & 17:42:00 & 17:44:00 & 113 \\
Borowska & DWORZEC AUTOBUSOWY & 17:44:00 & 17:47:00 & 113 \\
DWORZEC AUTOBUSOWY & DWORZEC GŁÓWNY & 17:47:00 & 17:50:00 & 113 \\
DWORZEC GŁÓWNY & Pułaskiego & 17:50:00 & 17:52:00 & 70 \\
Pułaskiego & Kościuszki & 17:52:00 & 17:54:00 & 70 \\
Kościuszki & Komuny Paryskiej & 17:55:00 & 17:56:00 & 70 \\
Komuny Paryskiej & pl. Wróblewskiego & 17:56:00 & 17:58:00 & 70 \\
pl. Wróblewskiego & Urząd Wojewódzki (Impart) & 17:58:00 & 18:01:00 & 70 \\
Urząd Wojewódzki (Impart & most Grunwaldzki & 18:01:00 & 18:02:00 & 70 \\
\hline
\end{tabular}
\end{adjustwidth}
% \end{center}

    \section*{Zadanie 1}
    Przeprowadzone zostało 100 testów przy użyciu alogrytmu Dijkstry dla różnych kryteriów. Testy zostały podzielone na 4 grupy po 25 testów, wg. ilości przystankówm przez które muszą przejechać. Odpowiednio:
    \begin{itemize}
        \item $A \rightarrow B$
        \item $A \rightarrow B \rightarrow C$
        \item $A \rightarrow B \rightarrow C \rightarrow D$
        \item $A \rightarrow B \rightarrow C \rightarrow D \rightarrow E$
    \end{itemize}
\end{document}
