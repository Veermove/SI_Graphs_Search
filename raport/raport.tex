\documentclass[12pt,a4paper]{article}
\usepackage[utf8]{inputenc}
\usepackage[T1]{fontenc}
\usepackage{amsfonts}
\usepackage{amsmath}
\usepackage{setspace}
\usepackage{wrapfig}
\usepackage[english]{babel}
\usepackage{graphicx}
\usepackage{caption}
\usepackage{amssymb}
\usepackage{changepage}
\usepackage{listings}

\title{Sztuczna inteligencja i inżynieria wiedzy, \\ Lista 1}
\author{Tymoteusz Trętowicz, 260451}
\date{}

\usepackage{xcolor}

\definecolor{codegreen}{rgb}{0,0.6,0}
\definecolor{codegray}{rgb}{0.5,0.5,0.5}
\definecolor{codepurple}{rgb}{0.58,0,0.82}
\definecolor{backcolour}{rgb}{0.95,0.95,0.92}

\lstdefinestyle{mystyle}{
    backgroundcolor=\color{backcolour},
    commentstyle=\color{codegreen},
    keywordstyle=\color{magenta},
    numberstyle=\tiny\color{codegray},
    stringstyle=\color{codepurple},
    basicstyle=\ttfamily\footnotesize,
    breakatwhitespace=false,
    breaklines=true,
    captionpos=b,
    keepspaces=true,
    numbers=left,
    numbersep=5pt,
    showspaces=false,
    showstringspaces=false,
    showtabs=false,
    tabsize=2
}

\lstset{style=mystyle}


\setstretch{1.1}

\begin{document}
    \maketitle
    \pagebreak
    \section*{Implementacjia}
    Implementacja grafu opiera się na słowniku nazw przystanków do ich wartości.
\begin{lstlisting}[language=Python]
class Node:
  def __init__(self, name, x, y):
    self.name = name
    self.x = float(x)
    self.y = float(y)
    self.edges = {}
\end{lstlisting}

    W każdym wierzchołku (obiekcie klasy Node) zapisujemy informacje o nazwie, wysokości i szerokości geograficznej, oraz o wychodzących z tego wierzchołka krawędziach (połączeniach do innych przystanków). \texttt{edges} jest strukturą danych w postaci zagnieżdżonych słowników: jest to mapa nazwy przystanku do mapy godzin odjazdu do tego przystanku do informacji o połączeniu. Przykładowo:
    \begin{lstlisting}[language=Python]

    graph['Rynek'].edges['Zamkowa']['12:52:30']
    \end{lstlisting}
    Jeżeli słownik obiektów kalsy Node nazywałby się \texttt{graph} to powyższe wyrażenie by zwróciło informacje o połączeniu z przystanku Rynek, do przystanku Zamkowa o godzinie 12:52:00. Te informacje to czas odjazdu z przystanku startowego, czas dojazdu do przystanku docelowego i nazwa lub numer lini.

    \pagebreak
    \subsubsection*{Algorytm Dijkstry}
    \noindent
    Zaimplementowany algorytm Dijkstry można przedstawić w postaci listy kroków:
    dla przystanku początekowego, przystanku docelowego i godziny startowej:
    \begin{enumerate}
        \item Utwórz pustą kolejkę przystanków do odwiedzenia.
        \item Dodaj przystanek początkowy, oraz godzinę znalezienia się na tym przystanku do kolejki.
        \item Pobierz \emph{obecny przystanek}, oraz ścieżkę z przystanku początekowego do obecnego przystanku z początku kolejki oraz usuń ten element z kolejki.
        \item Jeżeli \emph{obecny przystanek} jest przystnakiem końcowym zwróć scieżkę do obecnego przystanku.
        \item Dla każdego połączenia wychodzącego z tego przystanku: \begin{enumerate}
            \item Jeżeli godzina odjazdu do \textbf{następnego przystanku} już minęła, lub następny przystanek był już odwiedzony, pomiń to wychodzące połączenie.
            \item Policz koszt tego połączenia jako: $$c_{\text{dojazdu do \textbf{następnego przystanku}}} = c_{\text{doajzdu do \emph{obecnego przystanku}}} + c_{\text{połączenia}}$$
            \item Dodaj \textbf{następny przystanek} oraz to połączenie do kolejki przystnaków do odwiedzenia.
            \item Jezeli policzony koszt jest niższy niż obecny zapamiętany zapamiętaj nowy koszt jako najniższy oraz obecną scieżkę do \textbf{następnego przystanku}.
        \end{enumerate}
        \item Oznacz \emph{obecny przystanek} jako odwiedzony i wróć do punktu 3.
    \end{enumerate}
    Implementacja alogrytmu poza wymienionymi paramterami akceptuje jeszcze funkcję kosztu $f$ taką, że $$f(\text{obecny przystanek, następny przystanek, ...}) = c_{\text{połączenia}}$$:
    \section*{Zadanie 1}
    Przeprowadzone zostało 100 testów przy użyciu alogrytmu Dijkstry dla różnych kryteriów. Testy zostały podzielone na 4 grupy po 25 testów, wg. ilości przystankówm przez które muszą przejechać. Odpowiednio:
    \begin{itemize}
        \item $A \rightarrow B$
        \item $A \rightarrow B \rightarrow C$
        \item $A \rightarrow B \rightarrow C \rightarrow D$
        \item $A \rightarrow B \rightarrow C \rightarrow D \rightarrow E$
    \end{itemize}
    \subsection*{Kryteria}
    \subsubsection*{Kryetium czasu}
    Kryterium czasu jest rozumiane jako funkcja minimalizująca czas spędzony na przystanku czekając. Dla danego przystanku oraz odjeżdząjących z nich tramwajów (lub autobusów) jest ono liczone jako:$$c_{\text{połączenia}} = t_{\text{godzina odjazdu z przystanku}} - t_{\text{teraz}}$$.
    \subsubsection*{Kryetium przesiadek}
    Kryetium przesiadek jest rozumiane jako funkcja minimalizująca ilość przesiadek na połączeniu od przytanku startowego do przystanku końcowego mając nadal na uwadze kryetium czasu:
    \begin{align*}
        c_{\text{połączenia}}  &= 10 \cdot i_{\text{ilość przysiadek}} + 0.3 \cdot  c_{\text{kryetium czasu}}\\
        &= 10 \cdot i_{\text{ilość przysiadek}} + 0.3 \cdot  (t_{\text{godzina odjazdu z przystanku}} - t_{\text{teraz}})\\
    \end{align*}
    Kryterium przesiadek uwzględnia kryterium czasu z wagą $0.3$ (arbitralny wybór). Jeżeli waga dla kryterium czasu byłaby równa $1$, znaczyłoby to że przesiadka na inną linię jest równoważna z czekaniem 10 minut na przystanku.

\end{document}
