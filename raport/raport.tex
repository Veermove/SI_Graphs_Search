\documentclass[12pt,a4paper]{article}
\usepackage[utf8]{inputenc}
\usepackage[T1]{fontenc}
\usepackage{amsfonts}
\usepackage{amsmath}
\usepackage{setspace}
\usepackage{wrapfig}
\usepackage[english]{babel}
\usepackage{graphicx}
\usepackage{caption}
\usepackage{amssymb}
\usepackage{changepage}


\title{Sztuczna inteligencja i inżynieria wiedzy, \\ Lista 1}
\author{Tymoteusz Trętowicz, 260451}
\date{}
% \setcounter{tocdepth}{2}
% \renewcommand{\contentsname}{Spis treści}

\setstretch{1.1}

\begin{document}
    \maketitle
    \pagebreak
    \section*{Zadanie 1}
    Przeprowadzone zostało 100 testów przy użyciu alogrytmu Dijkstry dla różnych kryteriów. Testy zostały podzielone na 4 grupy po 25 testów, wg. ilości przystankówm przez które muszą przejechać.
    \begin{itemize}
        \item $A \rightarrow B$
        \item $A \rightarrow B \rightarrow C$
        \item $A \rightarrow B \rightarrow C \rightarrow D$
        \item $A \rightarrow B \rightarrow C \rightarrow D \rightarrow E$
    \end{itemize}
    \subsection*{Kryterium czasu}
    Kryterium czasu jest rozumiane jako funkcja minimalizująca czas spędzony na przystanku czekając. Dla danego przystanku oraz odjeżdząjących z nich tramwajów (lub autobusów) jest ono liczone jako:$$C = t_{\text{godzina odjazdu z przystanku}} - t_{\text{teraz}}$$.



\end{document}
